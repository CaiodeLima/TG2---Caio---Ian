% ----------------------------------------------------------
\chapter{Ambientes}
% ----------------------------------------------------------

Este modelo disponibiliza alguns ``ambientes'', ou seja, caixas de texto com formatação especial para certos tipos de elementos que são automaticamente numerados (e.g. \cref{prop:WYSIWYG}, \cref{thm:WYSIWYG}, etc.). Esses ambientes foram adaptados para o UnB\TeX\ a partir soluções utilizadas por \citeonline{Castro2019}.

\section{Exemplos de ambientes disponíveis}

\begin{definition}
O WYSIWYG (ou ``What You See Is What You Get - O que você vê é o formato final'') é um tipo de editor HTML que permite editar sua página da Web em uma visualização simplificada e sem código de aparência semelhante à do layout da página real.
\end{definition}

\begin{proposition}\label{prop:WYSIWYG}
    \LaTeX\ produz equações mais bonitas que qualquer editor WYSIWYG.
\end{proposition}

\begin{lemma}
    Teste
\end{lemma}

\begin{remark}
    \LaTeX\ produz equações mais bonitas que qualquer editor WYSIWYG.
\end{remark}

\begin{theorem}[Teorema LaTeX-WYSIWYG]\label{thm:WYSIWYG}
    Todo físico prefere usar código \LaTeX\ puro que qualquer editor WYSIWYG.
\end{theorem}

\begin{corollary}
    Teste
\end{corollary}

\begin{proof}
    Físicos gostam de equações bonitas. Editores What-You-See-Is-What-You-Get não são apropriados para fazer equações bonitas.\footnote{É certo que há editores WYSIWYG baseados em \LaTeX, mas eles não nos dão o mesmo nível de controle.} Logo, se algum físico preferisse usar um editor WYSIWYG no lugar de \LaTeX, não seria muito inteligente. Como todo físico é inteligente, o teorema está demonstrado \textit{ad absurdum}.
\end{proof}

\begin{example}
    Einstein usaria um editor WYSIWYG ou \LaTeX? \\
    Einstein era físico. Portanto, usando o teorema LaTeX-WYSIWYG, concluímos que ele usaria \LaTeX.
\end{example}