\chapter{Conclusão}
\label{conclusao}

O trabalho apresentado teve como objetivo o desenvolvimento de um sistema de classificação de queimadas no território da Amazônia Legal, por meio de um algoritmo de inteligência artificial para séries temporais. Ademais, também foi-se desenvolvido com o propósito de eventualmente ser integrado ao sistema do Painel do Fogo do Censipam.

A revisão literária indicou que embora este tipo de abordagem não seja inovadora no ramo do sensoriamento remoto em geral, é inovador no âmbito do sensoriamento dentro do território da Amazônia Legal.

O algoritmo desenvolvido é capaz de puxar dados oficiais e complementá-los, além de ser capaz de classificá-los com uma taxa de acurácia satisfatória.

Para trabalhos futuros, reserva-se a possibilidade de testar diferentes arquiteturas de redes neurais para séries temporais, como outras redes recorrentes ou redes convolucionais. Além disso, espera-se integrar o programa ao sistema oficial do Censipam como uma ferramenta classificadora de seus dados, podendo assim, serem utilizados por autoridades competentes no combate à incêndios florestais.





