% ----------------------------------------------------------
\chapter{Introdução}
\label{cap_intr}
% ----------------------------------------------------------

Este documento e seu código-fonte são exemplos de referência de uso da classe \textsf{unbtex}, uma customização da classe \textsf{abntex2} para a Universidade de Brasília (UnB). O documento exemplifica a elaboração de trabalho acadêmico (trabalho de conclusão de curso, dissertação e tese) a partir do UnB\TeX. O \abnTeX, por sua vez, é uma customização da classe \textsf{memoir} que visa atender os requisitos da norma ABNT NBR 14724:2011 \emph{Informação e documentação -- Trabalhos acadêmicos -- Apresentação}. Uma lista completa das normas observadas pelo \abnTeX\ é apresentada em \citeonline{abntex2classe}.

% Definição da nomenclatura que irá para a lista de siglas e abreviações
\nomenclature[A]{ABNT}{Associação Brasileira de Normas Técnicas}
\nomenclature[A]{UnB}{Universidade de Brasília}

O \abnTeX\ não é uma classe específica para nenhuma universidade ou instituição e implementa somente os requisitos das normas da ABNT. Sinta-se convidado a participar do projeto \abnTeX! Acesse o site do projeto em \url{http://www.abntex.net.br/}. Também fique livre para conhecer, estudar, alterar e redistribuir o trabalho do \abnTeX, desde que os arquivos modificados tenham seus nomes alterados e que os créditos sejam dados aos autores originais, nos termos da ``The \LaTeX\ Project Public License''\footnote{\url{http://www.latex-project.org/lppl.txt}}.

Encorajamos que sejam realizadas customizações específicas deste exemplo para universidades e outras instituições --- como capas, folha de aprovação, etc. Porém, recomendamos que ao invés de se alterar diretamente os arquivos do \abnTeX, distribua-se arquivos com as respectivas customizações, como feito no UnB\TeX. Isso permite que futuras versões do \abnTeX~não se tornem automaticamente incompatíveis com as customizações promovidas. Consulte \citeonline{abntex2-wiki-como-customizar} para mais informações.

Este documento deve ser utilizado como complemento do manual do \abnTeX\ \cite{abntex2classe} e da classe \textsf{memoir} \cite{memoir}. 

Parte das customizações feitas no \abnTeX\ são baseadas em soluções adotadas por \citeonline{Castro2019} para editoração dos livros da série \textit{Ensino de graduação} da Editora UnB.

%\begin{mdframed}[style=defnSty] % azul
\begin{mdframed}[style=plainSty] % verde

{\center \textsc{Texto motivador} \par}

\noindent Esperamos que o \abnTeX\ aprimore a qualidade do trabalho que você produzirá, de modo que o principal esforço seja concentrado no principal: na contribuição científica.
   
\end{mdframed}