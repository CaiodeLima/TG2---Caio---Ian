% ----------------------------------------------------------
\chapter{Trabalhos relacionados}
\label{relacionados}
% ----------------------------------------------------------


Muitos trabalhos abordam a detecção de incêndios florestais, especialmente usando aprendizado de máquina. Vários sensores podem ser usados, como
como câmeras IP \cite{forest-fire-ip-camera}, sensores sem fio para emissão de monóxido de carbono e temperatura \cite{forest-fire-detection-wireless-sensor} e dados de satélite \cite{queimadas_cerrado}.

\cite{Abid2021} fornece uma visão geral dos sistemas de detecção e previsão de incêndios florestais com base em algoritmos de aprendizado de máquina.
Os estudos que avaliam os fatores que afetam a ocorrência e o risco de incêndio também são discutidos, juntamente com as principais questões e resultados de cada estudo.

Estudos feitos para o bioma Cerrado mostram grande precisão no mapeamento de cicatrizes de queimaduras. O estudo de Arruda \cite{UNBVera} teve como objetivo fazer uma metodologia semiautomática para mapeamento de áreas queimadas no Cerrado, utilizando imagens Landsat e algoritmo Deep Learning nas plataformas Google Earth Engine e Google Cloud Storage. O estudo mostrou um potencial de dados de sensoriamento remoto com séries temporais Landsat utilizadas no projeto MapBiomas \cite{MapBiomasQueimadas}.

Os estudos para o bioma amazônico são mais escassos. No entanto, Lima \textit{et.al.} \cite{ModisGiovanna} realizou um trabalho focado no mapeamento de cicatrizes de queimaduras na Amazônia a partir de Modelo Linear de Mistura Espectral de imagens de sensores MODIS (MOD09). O estudo foi baseado no DETER (Projeto Detecção de Áreas Desmatadas em Tempo Real) \cite{deter} usando um algoritmo de classificação não supervisionado baseado em regiões. Os resultados mostraram que cerca de 50.000km² da superfície amazônica foram queimados e que os dados diários do sensor MODIS são uma importante fonte de informação para mapeamento de áreas queimadas.

Utilizando dados de sensoriamento remoto, Faria \textit{et.al.} \cite{severidade-do-fogo} desenvolveu um indicador quase em tempo real da severidade do fogo. São usadas informações dos satélites NOAA-20 e Suomi NPP para monitorar eventos de incêndio com base no nível de atenção que eles exigem e em quatro critérios: tamanho geral, duração, extensão e intensidade. A extensão espacial deste indicador inclui os biomas Amazônia e Pantanal brasileiros.

