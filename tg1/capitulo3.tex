% ----------------------------------------------------------
\chapter{Figuras e gráficos}
% ----------------------------------------------------------

\section{Introdução}

Sugiro que você guarde todas as figuras na pasta ``figuras'' para que seu projeto fique mais organizado. A \cref{fig:logoabntex} mostra como é fácil inserir uma figura com legenda.

\begin{figure}[htb]
	\centering
	\begin{minipage}{0.8\linewidth}
		\centering
		\includegraphics[width=\linewidth]{img-marca.pdf}
		\caption{Logo ABN\TeX2.} \label{fig:logoabntex}
	\end{minipage}
\end{figure}

Além de figuras, é possível inserir caixas de texto de diversos tipos, como axiomas, teoremas etc.

\section{Exercício}

\begin{exercise}
    Explique como Isaac Newton usaria cada um dos pacotes seguintes, se vivesse no tempo presente:
    \begin{enumerate}[label=(\Alph*)]
        \item Metapost
        \item TikZ
        \item PGFPlots
        \item PSTricks
    \end{enumerate}
\end{exercise}

\section{Algoritmo}

O \cref{alg:NNMPC} é um pseudo-código para obtenção de um controlador preditivo baseado em modelo e em redes neurais.

\begin{algorithm} [ht]
\caption{NN-based MPC pseudocode for consensus control}
\label{alg:NNMPC}
\begin{algorithmic}[1]

%\Procedure{Roy}{$a,b$}       \Comment{This is a test}
    \State System Initialization at $x(0), \theta(0)$
    \State $J = V(e(0),\theta(0))$
    \State $K^{ab}_0 = 0$ $\forall (a,b)$
    \State $dataset \gets [K_0,J]$
    \For{$k = 1:k_{max}$} \Comment{Simulation loop}
        \State $(W,Y) \gets train(dataset)$
        \State $\tilde{J} = Y\sigma(Wz)$ $\forall K^{ab}_k\pm \delta_K
        $
        \State $K_k^{\ast} = arg \min_{K_k}(\tilde{J})$
        \State $u(k) =  \left( L(\theta(k)) \otimes K_k^{\ast} \right) x(k)$
        \State $\hat{x}(k) = x(k)$; $\hat{\theta}(k) = \theta(k)$
        \For{$t = k+1:k+h+1$} \Comment{Prediction loop}
            %\For{$s = 1:S$} \Comment{Markov loop}
                \State $\hat{\theta}(t) = randMarkov(\hat{\theta}(t-1),\Pi)$
                \State $p = \hat{\theta}(t)$
                \State $\hat{u}(t) = \left( L(p) \otimes K_k^{\ast} \right) \hat{x}(t)$
                \State $\hat{x}(t+1) = F_p(\hat{x}(t))$
                \State $x_0(t) = mean(\hat{x}(t))$
                \State $\hat{e}(t) = F_p(\hat{x}(t)) - \boldsymbol{1_N} f_{p0}(x(t)) $
                \State $\hat{V}_t = \hat{e}^T(t)P_{p}\hat{e}(t)$
                \EndFor
                %\State $r(t) = r(t-k)\Pi^{k}$
                %\State $\bar{V}_t = \sum_{s=1}^S r_s(t) V_t^s$
                %\EndFor
        \State $\hat{J} = \sum_{t = k+1}^{k+h+1} \hat{V}_t$
        \State $dataset \gets [K^{\ast}_k,\hat{J}]$
        \State $x(k+1) = F_p(x(k))$
    \EndFor
    
%\EndProcedure

\end{algorithmic}
\end{algorithm}

\section{Programa}

\begin{lstlisting}
/**
* MSO: ativa o servo cujo eixo eh descrito
* por drive_axis; informacoes de controle
* sao gravadas em MSO_1
*/
  MSO(drive_axis,MSO_1);
/* Atribui o valor 0.0 ao primeiro elemento do array speed */
  speed[0] := 0.0; 
/* Atribui 1 para dataInitialized */
  dataInitialized := 1;
\end{lstlisting}

Para mais exemplos de programas, veja o \cref{apd:cdg}