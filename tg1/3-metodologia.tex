\chapter{Metodologia}
\label{metodologia}

A velocidade e a eficiência na detecção e monitoramento dos incêndios florestais são 
fundamentais para a viabilização do controle do fogo, e para isso é necessário possuir informações confiáveis da localidade e da área da queimada \cite{batista}. Visto a importância da preservação do território amazônico, este projeto se dispõe a desenvolver um sistema capaz de classificar por meio de algoritmos de aprendizado de máquina e de dados de satélites esses incêndios. Compondo um sistema inteligente de monitoramento para todo o território Amazônico.

De acordo com \cite{naqa}, “aprendizado de máquina é um ramo em evolução de algoritmos computacionais que são projetados para emular a inteligência humana aprendendo com o ambiente circundante."  Além disso, “pode melhorar automaticamente através da experiência” \cite{coogan}. Ao escolhermos desenvolver a ferramenta por meio de algoritmos de aprendizado de máquina, temos o objetivo de que ela possa fazer de maneira autônoma um trabalho de monitoramento que até então precisava da supervisão de um ser humano. % como fora feito anteriormente por muitos anos em projetos como o PRODES Analógico.

O trabalho aqui apresentado melhor se enquadra em um modelo de pesquisa experimental, onde diversas técnicas de \textit{machine learning}, processamento de dados, e engenharia de \textit{features} foram testadas com o objetivo de comparar a eficiência entre elas e como cada elemento é capaz de influenciar no desempenho do algoritmo.

Neste estudo, dados públicos disponíveis gratuitamente na internet foram utilizados como amostras e \textit{features} para o treinamento da rede. Os dados são disponibilizados por fontes confiáveis e renomadas, utilizadas mundialmente por pesquisadores e órgãos governamentais, como FIRMS e VIIRS, e classificado de acordo com o próprio Censipam e dados do GFED \textit{Amazon Dashboard}. Para treinamento foram utilizadas amostras e informações referentes ao ano de 2020, porém as fontes dos dados são capazes de liberar novos dados frequentemente de forma que o produto final sempre se mantenha atualizado com os conhecimentos mais recentes, podendo entregar suas previsões em tempo real sem que elas percam relevância com o passar do tempo.

É importante salientar que um diferencial dos dados oferecidos pelo Painel do Fogo é a capacidade de monitorar o que chamam de eventos de fogo, observando a evolução de um incêndio ao longo do tempo através de várias detecções de focos de incêndio em proximidade. Dessarte, se é possível analisar um incêndio florestal como um evento variante no tempo.

%\todo{mudar? a gente não quer fiscalizar de fato mas classificar apenas né, não identificar}

Como mencionado anteriormente, a qualidade e significância de cada dado dentro do projeto foi analisada experimentalmente. Normalmente ao trabalhar com \textit{machine learning} e redes neurais é possível medir o quanto que cada informação está sendo importante para as tomadas de decisões do modelo. Realizando tal analise foi-se capaz de definir quais informações deveríamos priorizar a obtenção, e quais não possuíam impacto, assim como quais modelos de aprendizado de máquina apresentavam melhor desempenho.

Com os resultados preliminares, a próxima etapa é investir em um modelo promissor e criar um código funcional que seja capaz de receber um banco de dados, povoá-lo com as \textit{features} necessárias e então retornar a saída de forma que ela possa ser incorporada de volta ao banco de dados com o máximo de independência.

A eficácia final do trabalho foi medida tomando como referência o padrão estabelecido pelo Censipam para tipificação de incêndios. Ele também está responsável pela revisão do produto e averiguação de sua utilidade.